\chapter{Background}
\label{background}
\section{Definitions}
This section outlines some terms used for convenience in this report.
\begin{description}
\item[Flash-card] A method of memorising facts where a physical card is written with a question on one side and the answer on the other.
\item[Review] A single review of a flash-card. The user is shown a vocabulary item in Japanese and must recall the meaning and pronounciation of the word. 
\item[Fact] A single vocabulary item with associated meaning and pronounciation.
\end{description}
\section{Machine Learning} \label{background_machinelearning}
When developing a computer algorithm which takes some data as an input and returns an output,
we will usually look at the underlying mechanism as to how we come to that conclusion. However it is not
always possible to know what is the underlying mechanism which produces an output
given certain inputs.

What makes machine learning algorithms unique is that they in a sense `learn'; given a large
enough set of observed inputs and outputs, a machine learning algorithm builds a model which infers
certain outputs given new inputs. We do not always need to know the underlying mechanism which
translates a set of inputs to a set of outputs, as long as the inputs and outputs are related and
a large enough data set is used.

What is `large enough'? This of course depends upon the problem domain, the number and type of inputs and outputs,
and how the inputs are related to the outputs. 

Various types of machine learning algorithms exist, however this report focuses on classification
algorithms -- that is algorithms which return an output as a `class' rather than as a continuous variables
(regression). Additionally, we will only be looking at supervised learning since training data will contain
outputs alongside inputs.

The following sections describe the machine learning techniques that were used in this report.

\subsection{Artificial Neural Network}
%Multilayer perceptrons are a type of artificial neural network 

\subsection{Support Vector Machines}
\section{The Forgetting Curve} \label{background_forgettingcurves}
The forgetting curve was first hypothesised by Hermann
Ebbinghaus\cite{ebbinghaus_memory:_1913} who observed that forgetting tends
to happen over time in an exponential fashion.

Ebbinghaus performed experiments on himself by attempting to memorise nonsense
syllables. He hypothesised that memory retention follows a curve similar to
equation \ref{eqn_forgetcurve} where $R$ is the retention of the information,
$t$ is time and $S$ is the relative strength of memory. This equation attempts to
estimate the rate at which a person forgets newly learned information by capturing
the exponential nature of forgetting which Ebbinghaus observed.

\begin{equation}
\label{eqn_forgetcurve}
R = e^{\frac{-t}{S}}
\end{equation}

The equation
is not intended to provide quantitative prediction of recall but rather to illustrate
the point that most of the `forgetting' happens soon after learning. Furthermore,
the equation illustrates that if the `strength of memory' $S$ can
be increased then the decay of the curve can be hampered.



\section{Spaced Repetition} \label{background_spacedrepetition}
Spaced repetition is a method for memorising pieces of information by reviewing
each piece of information at increasingly longer periods of time. It exploits the 
spacing effect of memory to improve efficiency in rote memorisation by attempting to 
have a student review a piece of
information \textit{just before} it is forgotten. 

Various studies have found that spacing out repetitions over time is
more effective than massed repetition or studying in a short space of time \cite{distributed_massed_2005}.
The type of spacing however is a more controversial topic, with some studies suggesting
fixed intervals are better \cite{fixed_intervals_2005}, while others suggest expanding intervals
are more effective \cite{effects_of_spacing_1963}. Regardless of this, spaced
repetition algorithms usually use expanding intervals in order to make study
time more efficient.

Spaced repetition can be used for memorising
nearly anything - equations, vocabulary, numbers, phrases, diagrams. A typical application
is using standard flash-cards, with a prompt to recall on one side and the correct answer on
the other. Depending on the how well the student recalls and the history of the flash-card, the
flash-card is rescheduled after each review. 

\begin{figure}[h!]
\includegraphics[width=11cm]{img/wired_forgetcurve.jpg}
\caption{Projected Forgetting Curves with Spaced Repetitions \cite{wolf_want_2008}}
\label{fig_projectedforgettingcurve}
\end{figure}

Figure \ref{fig_projectedforgettingcurve} shows an example of spaced repetition in
action. After initial memorisation, a student might revise the content when there
is a 90\% chance of recalling the content correctly -- which might be the following day.
After this first revision, the student is likely to remember the content for a longer
period of time and on day 10 will again have a 90\% chance of recall. This continues
for each subsquent revision, with each revision solidifying the content in the student's
mind and the chance of remembering diminishing at a slower rate over time.

As an added advantage, by prompting the user to recall and then to rate their answer -- the user
is in a sense being tested and thus actively engaging rather than passively `studying'.
Testing and the associated active retrieval has been shown to improve retention
significantly over passive study \cite{effects_of_recall_tests_1969},
\cite{retrieval_for_learning_2008}, \cite{power_of_testing_memory_2006}.

\begin{table}[h!]
\caption{Example of reviews for a single flash-card in spaced repetition}
\label{tbl_spacedrepetitionexample}
\begin{tabular}{|r|c|l|}
\hline
Review Date & Recall & New Interval \\
\hline
1 Jan & Incorrect & 0 days \\
1 Jan & Correct, difficult & 1 day \\
2 Jan & Correct, difficult & 2 days \\
4 Jan & Correct, easy & 5 days \\
9 Jan & Correct, easy & 12 days \\
21 Jan & Incorrect & 0 days \\
21 Jan & Correct, easy & 1 day \\
22 Jan & Correct, easy & 3 days \\
\ldots & \ldots & \ldots \\
\hline
\end{tabular}
\end{table}

Spaced repetition software automates this process by storing relevant data alongside each
flash-card in a database. The type of data stored depends upon the spaced repetition algorithm
used. Most algorithms store the current interval (in days) which represents the spacing,
an estimate of how long a student should be able to remember the word between reviews.
On each successful review, the interval is increased and the card rescheduled based on
the interval. Of course this is not an exact science, sometimes the student will not be able to recall,
and some algorithms take this into account and adjust based on the difficulty of the
particular word. An example of how a card might be rescheduled
is shown in table \ref{tbl_spacedrepetitionexample}.

Unfortunately very little publicly available research on spaced repetition
algorithms has been carried out. Dempster
postulates\cite{dempster_spacing_effect_1988} that it has simply not caught
on since it has not been known for very long, has not yet been demonstrated
satisfactorily with %TODO finish paragraph

\subsubsection*{SuperMemo 2 Algorithm (SM2)}
As one of the first spaced repetition applications available for personal computers, %TODO Ref
SuperMemo and its algorithms paved the way for other applications such as Mnemosyne and
Anki (see section \ref{background_similarprojects}). Developed by Piotr
Wozniack\cite{wozniak_optimization_1990}, the \textit{SuperMemo 2} algorithm was an
enhancement over the \textit{SuperMemo} algorithm primarily in that it would differentiate 
between items based on their difficulty to memorise\cite{wozniak_optimization_1990}.

The SuperMemo algorithm reschedules flash-cards a number of days into the future, known
as the \textit{interval}, where the interval is defined as the function $I(n)$ with $n$ the
current repetition number. For the first repetition, the interval is simply one day.
For the second repetition, the interval increases to six days. 
\begin{equation}
I(1) := 1
\end{equation}

\begin{equation}
I(2) := 6
\end{equation}

For all $n \geq 3$, equation \ref{eqn_sm2_3} applies.

\begin{equation}
\label{eqn_sm2_3}
I(n) := I(n-1) \times EF
\end{equation}

$EF$ is defined as the \textit{easiness factor} of the flash-card. The easiness factor
of the flash-card is adjusted on each review based on the answer given by the user with
equation \ref{eqn_sm2_4}

\begin{equation}
\label{eqn_sm2_4}
EF := EF + (0.1 - (5 - q) \times (0.08 + (5 - q) * 0.02))
\end{equation}

The easiness factor is bounded by the values $1.1 \leq EF \leq 2.5$ where 1.1 indicates
the most difficult flash-cards and 2.5 indicates the easiest. Before a user begins studying a
flash-card for the very first time, the associated easiness factor is set to 2.5.

\section{Similar Projects} \label{background_similarprojects}
\subsection*{Memrise}
Memrise is a private company which produces web-based flashcard software.

\subsection*{The Mnemosyne Project}
Mnemosyne is open source spaced repetition software collecting anonymised data from its many users in order
to evaluate the effectiveness of the implemented spaced repetition algorithm \cite{peter_bienstman_principles_2012}.
Mnemosyne uses a modified version of the Supermemo algorithm.
The project does not appear to have produced any papers or research publications at this time.

\begin{figure}[h!]
\includegraphics[width=4cm]{img/mnemosyne_screen.png}
\caption{Screenshot of Mnemosyne in use}
\end{figure}

\subsection*{Anki}
Anki is one of the most full-featured open-source spaced repetition applications available.
Anki allows users to attach images, sounds, and even embed \LaTeX \ equations in flash-cards. 

The developer of Anki decided against SuperMemo 3 and later algorithms instead opting for
the SuperMemo 2 algorithm because of the complexity that the SM3+ algorithms introduce\cite{anki_faq}.
