\chapter*{Introduction}

With the computing power available today, analysis of very large amounts of data is providing
new insights in many areas that were previously impossible. One of these areas is
education and the possibilities for analysing learning behaviour, patterns and memory
could potentially lead to many improvements and efficiencies in the way education is conducted.

This thesis project focuses on self motivated rote memorisation of foreign vocabulary in
a University setting, with the use of a flash-card like online learning
environment to record information about student study. The goal is to analyse the data
recorded from the study of the foreign vocabulary and subsequently predict whether a student will
correctly recall the foreign pronounciation and the meaning in their native language
at a future point in time.

The report outlines some background behind the memory and the effect of spacing out reviews,
and the subsequent development of spaced repetition algorithms used to track study
and improve efficiency of study. The report then outlines the process to build the
online learning environment which records usage by University students as they
study foreign vocabulary using the SuperMemo 2 spaced repetition algorithm. The spaced
repetition algorithm keeps track of parameters for each individual vocabulary item for
each user which are then recorded in review data.

Analysis is performed on the review data including the use of machine learning algorithms
attempting to predict the recall of a vocabulary item given only the known spaced repetition parameters
of a word for a particular user. The same review data is also grouped to produce curves
which illustrate the rate of forgetting a vocabulary item.
