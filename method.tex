The system will be deployed for use by students enrolled in a Japanese course at
The University of Queensland.

Users will be randomly assigned to one of two groups - Group A or Group B. Group
A will use a learning environment with a series of learning modules and tests,
while Group B will use the same learning environment with the addition of some 
game mechanics. At no point will users be required to carry out any task for
assessment or otherwise, and users may opt-out and delete all their data at
any time.

The actions of the students will be logged over a six-week period during which
the system will be available to them. Specifically which actions will be logged
is discussed in section \ref{scope_datacollection}. These actions will be 
analysed to determine the engagement of the users in each group.

In an attempt to minimise the variables present, the number of game mechanics
used has been reduced to a few easily quantifiable concepts:
\begin{itemize}
	\item Challenges - completing a challenge unlocks new challenges
	\item Progress Indication - feedback on progress through a challenge as well
		as feedback on total learning progress through the environment
	\item Scoring - users earn points on successful completion of challenges. To
		make scores meaningful, users will see their location on a leaderboard
		which ranks them against other users
\end{itemize}
