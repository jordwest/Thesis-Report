\section{Methods and Materials}
\subsection{Ethical Clearance}
\subsubsection{Data Considerations}
\subsubsection{Application Details}

\subsection{Software Design}
\subsubsection{Requirements}
\subsubsection{Tools}
\paragraph{Github}
\paragraph{Ruby on Rails}
\paragraph{Heroku}
\paragraph{Highcharts}
\paragraph{Twitter Bootstrap}
(\url{http://twitter.github.com/bootstrap})

Twitter Bootstrap is a set of default styles for websites and web applications, provided as open-source by Twitter. Using Twitter Bootstrap rapidly speeds up theming of a web application with default looks for navigation, buttons, text and layout.

Compare the following pages with and without Twitter Bootstrap default styles added:
%\figure{}

More significantly, Twitter Bootstrap offers a 'responsive' layout system which provides a reduced screen size (ie. smartphone) layout with little to no extra work on the part of the developer. This means a smartphone version of the web application could be designed at the same time. Twitter Bootstrap was chosen for this reason.
\subsubsection{Screen Mockups}
\subsubsection{Spaced Repetition Algorithm}
\subsubsection{Data storage, formatting and output}

\subsection{Data Analysis and Prediction}
\subsubsection{R Programming Language}
\subsubsection{Usage Data}
\subsubsection{Forgetting Curves}
\subsubsection{Prediction of Recall}